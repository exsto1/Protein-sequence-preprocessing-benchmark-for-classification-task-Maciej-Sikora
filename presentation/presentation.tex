\documentclass[10pt]{beamer}
\usetheme{Warsaw}

\setbeamertemplate{footline}{}
\setbeamersize{text margin left=3.5mm,text margin right=3.5mm} 

\usepackage{hyperref}
\hypersetup{
    colorlinks=true,
    linkcolor=blue,
    filecolor=magenta,      
    urlcolor=blue,
}
 
\urlstyle{same}

\usepackage{datetime}
\newdateformat{specialdate}{\twodigit{\THEDAY}.\twodigit{\THEMONTH}.\THEYEAR}

\usepackage[utf8]{inputenc}

\usepackage{float}
\usepackage{graphicx}
\graphicspath{ {./images/} }

\usepackage{verbatim}

\usepackage{caption}
\captionsetup[figure]{name=Fig.}


\title{Benchmark comparison of protein sequence preprocessing effect on learning task for Pfam family classification}
\author{Maciej Sikora}

\begin{document}
\frame{\titlepage}


\begin{frame}
\frametitle{Inspiration and problem overiew}
\begin{itemize}
\item \textbf{Data preprocessing for size reduction without loosing crucial information.}
\end{itemize}

\begin{itemize}
\item Growing amount of biological data
\item Uniprot
\end{itemize}
\end{frame}

\begin{frame}
\frametitle{Inspiration and problem overiew}
\begin{itemize}
\item \textbf{Data preprocessing for size reduction without loosing crucial information.}
\end{itemize}

\begin{itemize}
\item Most data is unreviewed
\item Automated data annotating
\item Need for better models
\item More complicated models take longer to train
\end{itemize}
\end{frame}

\begin{frame}
\frametitle{Compared methods of preprocessing}
\begin{itemize}
\item Data source: Swissprot -- manually reviewed part of the Uniprot database
\item Filtering by most frequent organisms and families
\item CD-HIT -- removing very simmilar protein sequences
\item Padding
\item Splitting to stratified train and test pools
\item Shuffling
\end{itemize}

\begin{itemize}
\item Original
\item Singletons -- dtype int8
\item Triplets -- dtype int16
\item Biovec
\end{itemize}

\end{frame}

\begin{frame}
\frametitle{Tested models}
\begin{itemize}
\item Decision trees
\item Random trees
\item MLP
\item Nearest neighbours
\item Machine Learnig (Simple Dense model)
\end{itemize}

\begin{itemize}
\item Grid Search for optimal parameters
\item Cross-validation
\end{itemize}
\end{frame}

\begin{frame}
\frametitle{Results}
\end{frame}

\begin{frame}
\frametitle{Summary}
\begin{itemize}
\item Treating protein data as string objects might have a negative impact on the training process.
\item Simple conversion to numerical data improves both quality and runtime.
\item Further numerical categorization for triplets while speeding up the learning process is loosing some of the information leading to decreased accuracy.
\item Neighbourhood of aminoacids can be analysed using more sophisticated biovec model.
\item Biovec yields both better results and over 500\% faster time compared to the original model.
\item Biovec model data weights over 20 times less than the original one.
\end{itemize}
\end{frame}
\end{document}
